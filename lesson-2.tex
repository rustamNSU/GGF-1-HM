\documentclass[10pt]{beamer}
\usepackage[T2A]{fontenc}
\usepackage[utf8]{inputenc}
\usepackage[english,russian]{babel}

\usepackage{amssymb,amsfonts,amsmath}
\usepackage{cite,enumerate,float,indentfirst}
\usepackage{graphicx}
\usepackage{hyperref}
\hypersetup{
    colorlinks=true,
    linkcolor=blue,
    filecolor=magenta,      
    urlcolor=cyan,
}
\numberwithin{equation}{subsection}

% \hypersetup{pdfstartview=FitH,  linkcolor=linkcolor,urlcolor=urlcolor, colorlinks=true}
\hypersetup{unicode=true}

\graphicspath{{images/}}
\DeclareGraphicsExtensions{.pdf,.png,.jpg}
\usepackage{graphicx}
\usepackage{colortbl}
\usepackage{xcolor}
\usepackage{ifthen}
\usepackage{subfigure}
\usepackage{amsthm}
\usepackage{listings} % code format pasting
\lstset{language=Mathematica}
\lstset{basicstyle={\sffamily\scriptsize},
    numbers=left,
    numberstyle=\tiny\color{gray},
    numbersep=5pt,
    breaklines=true,
    captionpos={t},
    frame={lines},
    rulecolor=\color{black},
    framerule=0.5pt,
    columns=flexible,
    tabsize=2
}

% Some themes
\usetheme{Hannover}
% \usetheme{Madrid}
\usefonttheme{professionalfonts}

\setbeamercolor{block title}{use=structure,fg=white,bg=structure.fg!75!black}
\setbeamercolor{block body}{parent=normal text,use=block title,bg=block title.bg!10!bg}

\title[]{Понятие множества. Понятие функции. Последовательности.}
\author[]{Абдуллин Рустам Фаритович}
\institute[НГУ]
{
    \vspace{0.5cm}
    \begin{minipage}{0.6\linewidth}
        \begin{center}
            \scriptsize
            \textbf{ НОВОСИБИРСКИЙ ГОСУДАРСТВЕННЫЙ УНИВЕРСИТЕТ, НГУ}
        \end{center}
    \end{minipage}
}
\date{\today}

\setbeamertemplate{navigation symbols}{}
\begin{document}
    \begin{frame}
        \titlepage
    \end{frame}

    \section*{Содержание}
    \begin{frame}
        \tableofcontents
    \end{frame}

    \section{Теория множеств}
    \begin{frame}
        \frametitle{Предпосылки}
        \begin{itemize}
            \item \href{https://en.wikipedia.org/wiki/Set_theory}{Наивная теория множеств} -- Георг Кантор \\
                \begin{enumerate}
                    \item парадокс Кантора
                    \item парадокс Рассела
                \end{enumerate}
            \item Аксиоматические теории множеств (Цермело, Френкель, Борель, Лебег)
        \end{itemize}
    \end{frame}

    \begin{frame}
        \frametitle{Основные понятия}
        \begin{itemize}
            \item Множество -- набор (совокупность) каких-либо объектов, которые называются элементами данного множества. Принадлежность объекта к множеству обозначается
            $ x \in A$ ($x$ -- элемент множества $A$).
            \item Множество, которое не имеет элементов, называется пустым $\varnothing $.
            \item Множество $A$ -- подмножество множества $B$ ($A \subseteq B$), если для любого элемента $x \in A$ верно $x \in B$.
            \item $A$ -- строгое подмножество $B$ ($A \subset B$), если $A \subseteq B$ и существует $y \in B$, такой что $y \notin A$.
        \end{itemize}
    \end{frame}

    \begin{frame}[fragile]
        \frametitle{Основные операции над множествами}
        \begin{itemize}
            \item \textbf{объединение}, обозначается как $A \cup B$, содержит все элементы из $A$ и $B$ 
            \item \textbf{пересечение}, обозначается как $A \cap B$, содержит элементы, содержащиеся и в $A$, и в $B$
            \item \textbf{разность}, обозначается как $A \setminus B$, содержит все элементы из $A$, которые не входят в $B$
        \end{itemize}
        \begin{block}{Теорема 1}
            \phantomsection
            \label{th:th1}
            Если верно $A \subseteq M$ и $B \subseteq M$, то $(M \setminus A) \cup (M \setminus B) = M \setminus (A \cap B)$ и $(M \setminus A) \cap (M \setminus B) = M \setminus (A \cup B)$
        \end{block}
    \end{frame}

    \section{Отображение (функция)}
    \begin{frame}
        \frametitle{Основные определения}
        см. \textit{Лекции + рукопись}
        \begin{itemize}
            \item Понятие функции
            \item Область определения (область существования) функции
            \item Область значений функции
            \item Образ функции (\textit{в чем разница с областью значений})
        \end{itemize}
    \end{frame}

    \section{Последова-тельность}
    \begin{frame}
        \frametitle{Последовательность}
        \begin{itemize}
            \item \textit{Последовательность} -- пронумерованный набор объектов (допускаются повторения)
            \item \textbf{Последовательность} -- всякое отображение $f: \mathbb{N} \rightarrow X$
            \item \textbf{Числовая последовательность} -- последовательность, где $X = \mathbb{R}$ ($\{x_n\}_{n=1}^\infty$)
        \end{itemize}
    \end{frame}

    \begin{frame}[fragile]
        \frametitle{Предел последовательности}
        \begin{block}{Определение 1}
            \small
            Число $a$ -- является пределом числовой последовательности $\{x_n\}_{n=1}^\infty$, обозначается $\displaystyle\lim_{n \rightarrow \infty} x_n = a$, 
            если $\forall \varepsilon > 0 \; \exists N \in \mathbb{N}\;|\;\forall n > N$ верно $\left| x_n - a \right| < \varepsilon $.
        \end{block} 
        \begin{block}{Определение 2}
            \small
            Если для элементов последовательностей $\{x_n\}_{n=1}^\infty$, $\{a_n\}_{n=1}^\infty$, $\{b_n\}_{n=1}^\infty$ верно $a_n \leq x_n \leq b_n$ и 
            $\displaystyle\lim_{n \rightarrow \infty} a_n = \lim_{n \rightarrow \infty} b_n = c$, то $\lim_{n \rightarrow \infty} x_n =c$
        \end{block} 
    \end{frame}

    \begin{frame}
        \frametitle{Примеры}
        \begin{itemize}
            \item $\displaystyle\lim_{n \rightarrow \infty} n = \infty $;
            \item $\displaystyle\lim_{n \rightarrow \infty} \frac{1}{n} = 0$;
            \item $\displaystyle\lim_{n \rightarrow \infty} \frac{x^m+1}{x^n+1} = 0 $, если $m<n$;
            \item $\displaystyle\lim_{n \rightarrow \infty} \left(1+\frac{1}{n}\right)^n = e $.
        \end{itemize}
    \end{frame}

    \section{Теория пределов. Непрерывность.}

    \begin{frame}[fragile]
        \frametitle{Определения}
        \begin{block}{Предел функции в точке}
            \footnotesize
            $\displaystyle\lim_{x \rightarrow a} f(x) = A$, если $\forall \varepsilon>0 \; \exists \delta = \delta(\varepsilon)$, 
            такое что для всех $x$ из области определения функции и удовлетворяющих условию $0<|x-a|<\delta$ выполняется
            $|f(x)-A| < \varepsilon$.
        \end{block}
        \begin{block}{Предел слева}
            \footnotesize
            $\displaystyle\lim_{x \rightarrow a-0} f(x) = A^\prime$, если $\forall \varepsilon>0 \; \exists \delta = \delta(\varepsilon)$, 
            такое что для всех $x$ из области определения функции и удовлетворяющих условию $0<a-x<\delta$ (или тоже самое $(a-\delta<x<a)$) выполняется
            $|f(x)-A^\prime| < \varepsilon$.
        \end{block}
        \begin{block}{Предел справа}
            \footnotesize
            $\displaystyle\lim_{x \rightarrow a+0} f(x) = A^{\prime\prime}$, если $\forall \varepsilon>0 \; \exists \delta = \delta(\varepsilon)$, 
            такое что для всех $x$ из области определения функции и удовлетворяющих условию $0<x-a<\delta$ (или тоже самое $(a<x<a+\delta)$) выполняется
            $|f(x)-A^{\prime\prime}| < \varepsilon$.
        \end{block}
    \end{frame}

    \begin{frame}
        \frametitle{Непрерывность}
        Предел функции существует, тогда и только тогда, когда $A^\prime = A^{\prime\prime}$
        Виды разрыва в точке
        \begin{itemize}
            \item \textbf{Устранимый}, если $A^\prime = A^{\prime\prime}$
            \item \textbf{Разрыв первого рода}, если $A^\prime \neq  A^{\prime\prime}$ и $A^\prime - A^{\prime\prime} < \infty$
            \item \textbf{Разрыв второго рода}, если $A^\prime \neq  A^{\prime\prime}$ и хотя бы один из пределов равен $\infty$
        \end{itemize}
    \end{frame}

    \begin{frame}
        \frametitle{Примеры}
        \begin{itemize}
            \item Если функция непрерывна в точке, то предел функции к этой точке равен значению функции в этой точке
            \item $\displaystyle f(x) = \frac{x^2 - 1}{x-1}$ в точке $x=1$
            \item $f(x) = \operatorname{sgn}(x)$
            \item $\displaystyle f(x) = \frac{1}{x}$
            \item замечательные пределы:
            \begin{enumerate}
                \item $\displaystyle\lim_{x \rightarrow 0} \frac{\sin x}{x} = 1$
                \item $\displaystyle\lim _{{x\to \infty }}\left(1+{\frac  {1}{x}}\right)^{x}=e$.
            \end{enumerate}
        \end{itemize}
    \end{frame}

    \section{Homework}
    \begin{frame}
        \frametitle{Homework}
        \begin{itemize}
            \item \textit{Демидович. Сборник задач и упражнений по математическому анализу.} № 1, 2, 3, 4, 6, 7, 151, 152, 160, 166, 168, 175
            \item Доказать \hyperlink{th:th1}{теорему 1}
        \end{itemize}
    \end{frame}

\end{document}